% SHIFT - project abstract

\documentclass{sig-alternate}

\usepackage{url}
\usepackage{color}
\usepackage{enumerate}
\usepackage{balance}
\usepackage{verbatim}
\usepackage{enumitem}
\permission{}
\CopyrightYear{2012}
%\crdata{0-00000-00-0/00/00}
\begin{document}

\title{ABLS - An Attribute Based Logging System for the Cloud}
\numberofauthors{1}
\author{
\alignauthor{
Christopher A. Wood \\
Department of Computer Science \\
caw4567@rit.edu
}}
\date{\today}
\maketitle
\begin{abstract}

User-based non-repudiation is a system security property that provides indisputable evidence that links
specific actions to individual users (or entities) that trigger such actions. Cryptographically speaking, non-repudiation requires that the integrity and 
origin of all data should be provable. In essence, this enables system audits to be conducted that can
identify data misuse (and thus, potential security policy violations) by comparing the sources of system events
with all entities authorized to invoke these events. Therefore, treating non-repudiation as a required system 
quality attribute in the SaaS architecture is likely to become a common trend in the commercial, 
government, and even more specifically, the health-care domain.

System audits typically use log files to determine the cause and effect of events that took 
place during the system's lifetime. In order to provide accurate information for non-repudiation purposes,
it is often necessary to place some amount user-sensitive data in these log files that can be used
to trace data back to its origin. As such, logs of events generated by a client that is being served must
maintain data confidentiality and integrity should the system be compromised. These goals are commonly achieved
using a combination of encryption and signature techniques \cite{Ma2008-FssAgg}. However, traditional approaches to
encryption and signature generation and verification are becoming less effective in the context of cloud
applications. Furthermore, naive approaches to log security that are based on tamper-resistant hardware and 
maintaining continuous secure communication channels between a log aggregator and end user are no longer useful in 
the context of cloud-based applications \cite{Schneier1999-Secure}. 

Symmetric-key and public-key encryption of log entries are very common confidentiality techniques 
proposed in the literature. However, in cloud-based applications, these schemes are becoming less useful.
There is a need for a robust access control mechanism that enables dynamic user addition and revocation
with minimal overhead (i.e. re-encrypting a subset of the log database should be avoided). Both symmetric- and 
public-key cryptosystems lack in that access policies must be tied directly to keys used for
encryption and decryption. If the access policy for a set of log messages needs to be changed, then both the keys used to
encrypt and decrypt such log entries will need to be regenerated and distributed, and the entries must also
be re-encrypted. Both of these tasks can be very expensive. 

In addition, symmetric-key cryptosystems require keys to be shared among users who need access to the 
same set of logs, which requires a secure and comprehensive key management and distribution policy. 
From a storage perspective, public-key cryptosystems (e.g. RSA and ElGamal) suffer from the extra data transfer and 
storage requirements for large cryptographic keys and certificates. There may be insufficient resources 
to maintain a public-key infrastructure (PKI) for managing keys and digital certificates for all users. 

In terms of log file integrity, aggregate signature schemes that support forward secrecy through the use of 
symmetric- and public-key cryptosystems are also becoming outdated \cite{Yavuz2009-BAF}. Symmetric-key schemes may promote
high computational efficiency for signature generation, but they do not directly support public verifiability for 
administrators and auditors. This means that robust key distribution schemes or the introduction of a trusted third party (TTP) are needed to ensure all required parties can access the necessary log information. Such schemes also 
suffer from high storage requirements and communication overhead. Public-key schemes have similar issues, 
as the increased key size leads to even larger storage requirements and less computational efficiency. 

Collectively, we see that a balance between encryption and signature generation and verification performance is
needed to support the unique scalability and resource usage requirements for cloud-based applications.
Attribute-based encryption (ABE), a new cryptographic scheme that uses user attributes (or roles, in certain 
circumstances) to maintain the confidentiality of user-sensitive data, has an appealing application
to logging systems maintained in the cloud and is capable of satisfying the aforementioned confidentiality 
requirements. In addition, authenticated hash-chains have been shown to be effective at enforcing log file integrity
in numerous logging schemes  \cite{Schneier1999-Secure}. 

ABLS, an attribute-based logging system that supports ciphertext-policy attribute-based encryption (CP-ABE) \cite{Bethencourt2007-CPABE} 
and authenticated hash-chain constructions for log file confidentiality and integrity, respectively, was recently designed and implemented
by the primary author.  The preliminary ABLS architecture and design was weak with regards to the 
scalability of encryption operations under heavy traffic loads,
the relational database schema to store log-data and other sensitive information, and the interactions 
with database servers. To address these issues,
we propose a set of extensions that modify ABLS in the following ways:
\begin{enumerate}
	\item The CP-ABE encryption scheme will be replaced with a hybrid cryptosystem in which the Advanced
	Encryption Standard (AES), the standardized symmetric-key encryption algorithm, is used to encrypt user
	session data by a key that is protected with CP-ABE encryption. In the event that a user generates log messages
	with varying sensitivity levels during a single session, multiple symmetric keys will be produced to maintain
	the confidentiality of information in different security classes.
	\item The relational database storage scheme will either be changed to include a masking column in the appropriate
	log table to hide user identities or replaced entirely with a document- or object-based database. Both of these
	modern database systems are similar in semantics (if documents are conceptually treated as serialized objects),
	so a product and literature survey will be conducted prior to selecting an adequate replacement that satisfies
	the necessary security and performance requirements. Though, based on preferences and architectural experience,
	a document-based database such as MongoDB will be the likely candidate. Also, since the ABLS architecture is highly
	structured around a relational schema to store data, this change will require modifications made in the logging, attribute authority, 
	and auditing modules in order to maintain functional correctness. 
	\item Database security mechanisms, including fine-grained access policies for protected databases and encryption of data-in-transit, 
	will be implemented. This is particularly
	important for the databases that store cryptographic keys. Also, since preference is given to document-based databases such as MongoDB,
	third-party services such as zNcrypt (provided by Gazzang) that are specifically tailored to this DBMS will likely be used to enforce 
	access control to sensitive databases.
	\item The auditing module will be extended to include automated audits of database operations, including changes
	to the database structure and both successful and unsuccessful client connections with the database. Currently,
	the auditing module is only designed (and partially implemented) to support strategy-based audits on database contents. However,
	in this type of application, database performance and security are just as critical. Thus, with this change, there will be two audit 
	techniques that can roughly be equated to data and policy inspections, both of which will only be accessible to users with the
	appropriate privileges. This access control will be likely be enforced using username and password credentials.
\end{enumerate}

Altogether, the security features of database authentication, encryption, and auditing will be further expanded upon
in the existing ABLS architecture. With these changes, we will then re-evaluate the ABLS at an architectural, security, 
and performance perspective to determine its usefulness in cloud-based settings. 

\end{abstract}

\begin{comment}
\begin{table}
\centering
\caption{Feelings about Issues}
\begin{tabular}{|l|r|l|} \hline
Flavor&Percentage&Comments\\ \hline
Issue 1 &  10\% & Loved it a lot\\ \hline
Issue 2 &  20\% & Disliked it immensely\\ \hline
Issue 3 &  30\% & Didn't care one bit\\ \hline
Issue 4 &  40\% & Duh?\\ \hline
\end{tabular}
\end{table}

\begin{figure}[htb]
\label{sample graphic}
\begin{center}
\includegraphics[width=1.5in]{fly.jpg}
\caption{A sample black \& white graphic (JPG).}
\end{center}
\end{figure}
\end{comment}

\bibliographystyle{abbrv}
\bibliography{../abls}
% You must have a proper ".bib" file
%  and remember to run:
% latex bibtex latex latex
% to resolve all references
\balance
\end{document}
