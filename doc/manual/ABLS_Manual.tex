\documentclass[a4paper,11pt]{report}
%
%--------------------   start of the 'preamble'
%
\usepackage{graphicx,amssymb,amstext,amsmath,caw,verbatim,dirtree,listings}
%

%%    homebrew commands -- to save typing
\newcommand\etc{\textsl{etc}}
\newcommand\eg{\textsl{eg.}\ }
\newcommand\etal{\textsl{et al.}}
\newcommand\Quote[1]{\lq\textsl{#1}\rq}
\newcommand\fr[2]{{\textstyle\frac{#1}{#2}}}
\newcommand\miktex{\textsl{MikTeX}}
\newcommand\comp{\textsl{The Companion}}
\newcommand\nss{\textsl{Not so Short}}

%
%---------------------   end of the 'preamble'
%
\begin{document}
%-----------------------------------------------------------
\title{\textbf{ABLS} \\ An Attribute-Based Logging System for the Cloud {\small User Manual}}
\author{Christopher A. Wood \\ {\tt caw4567@rit.edu}}
\maketitle
%-----------------------------------------------------------
\begin{abstract}\centering
User-based non-repudiation is an increasingly important property of cloud-based applica-tions. It provides irrefutable evidence that ties system behavior to specific users, thus enabling strict enforcement of organizational security policies. System logs are typically used as the basis for this property. Thus, the effectiveness of system audits based on log files reduces to the problem of maintaining the integrity and confidentiality of log files. In this project, we study the problem of building secure log files. We investigate the benefits of ciphertext-policy attribute-based encryption (CP-ABE) to solve a variety of log design issues. In addition, we also present the architecture and a preliminary analysis for a proof-of-concept system that fulfills the confidentiality and integrity requirements for a secure log.
\end{abstract}
%-----------------------------------------------------------
\tableofcontents
%-----------------------------------------------------------
\chapter{Bootstrapping}

ABLS comes packaged with a set of configuration scripts and SQL files that initialize
the database in a clean state. These files are included in the DatabaseModule directory
that comes packaged with ABLS, as shown below:

\dirtree{%
.1 ABLS.
.2 main - main executable.
.2 LoggerModule.
.2 PolicyEngineModule.
.2 AuditModule.
.2 Common.
.2 TestModule.
.2 DatabaseModule.
.3 bootstrap - bash script.
.3 bootstrap SQL files.
}

In order to configure an new ABLS instance to be run on a server in development mode, one must run
the following commands from the root ABLS directory.

\begin{lstlisting}
$> ./DatabaseModule/bootstrap 
$> python main.py -c 
\end{lstlisting}

The first bootstrap script will wipe the database files and configure them for use.
This is the script that should be modified if the user wants to change the physical
location of each database server. The second command will tell the main ABLS executable
script to ``configure'' the database with some fake data for testing purposes. As such,
this should only be used when configuring ABLS for development and testing purposes.

Once complete, the user should then run the following command from the root ABLS directory.

\begin{lstlisting}
$> python main.py -s
\end{lstlisting}

This command will ``start'' an ABLS instance on the local host. If one wants to deploy 
an ABLS instance in production mode, they will only run the main executable with the 
``-s'' flag, not the ``-c'' flag. Also, these flags can be combined during the bootstrapping
process to save time, as shown below.

\begin{lstlisting}
$> ./DatabaseModule/bootstrap 
$> python main.py -c -s
\end{lstlisting}
%-----------------------------------------------------------
%\addcontentsline{toc}{chapter}{\numberline{}Bibliography}
%\include{biblio}
%-----------------------------------------------------------
%\appendix
%\include{app4}
%\include{app1}
%\include{app2}
%\include{app5}
%\include{app3}
%-----------------------------------------------------------

\bibliography{ref}{}
\bibliographystyle{plain}

\end{document}
