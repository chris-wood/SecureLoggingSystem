\documentclass[a4paper,11pt]{report}
%
%--------------------   start of the 'preamble'
%
\usepackage{graphicx,amssymb,amstext,amsmath,caw,verbatim,dirtree,listings}
%

%%    homebrew commands -- to save typing
\newcommand\etc{\textsl{etc}}
\newcommand\eg{\textsl{eg.}\ }
\newcommand\etal{\textsl{et al.}}
\newcommand\Quote[1]{\lq\textsl{#1}\rq}
\newcommand\fr[2]{{\textstyle\frac{#1}{#2}}}
\newcommand\miktex{\textsl{MikTeX}}
\newcommand\comp{\textsl{The Companion}}
\newcommand\nss{\textsl{Not so Short}}

%
%---------------------   end of the 'preamble'
%
\begin{document}
%-----------------------------------------------------------
\title{\textbf{ABLS} \\ An Attribute-Based Logging System for the Cloud {\small User Manual}}
\author{Christopher A. Wood \\ {\tt caw4567@rit.edu}}
\maketitle
%-----------------------------------------------------------
\begin{abstract}\centering
User-based non-repudiation is an increasingly important property of cloud-based applica-tions. It provides irrefutable evidence that ties system behavior to specific users, thus enabling strict enforcement of organizational security policies. System logs are typically used as the basis for this property. Thus, the effectiveness of system audits based on log files reduces to the problem of maintaining the integrity and confidentiality of log files. In this project, we study the problem of building secure log files. We investigate the benefits of ciphertext-policy attribute-based encryption (CP-ABE) to solve a variety of log design issues. In addition, we also present the architecture and a preliminary analysis for a proof-of-concept system that fulfills the confidentiality and integrity requirements for a secure log.
\end{abstract}
%-----------------------------------------------------------
\tableofcontents
%-----------------------------------------------------------
\section{Bootstrapping}

ABLS comes packaged with a set of configuration scripts and SQL files that initialize
the database to a clean state. These files are included in the DatabaseModule directory
that comes packaged with ABLS, as shown below:

\dirtree{%
.1 ABLS.
.2 Main.py - main executable.
.2 Bootstrap.py - bootstrap file for the database.
.2 LoggerModule.
.2 PolicyEngineModule.
.2 AuditModule.
.2 VerifyModule.
.2 CryptoModule.
.2 Common.
.2 TestModule.
.2 DatabaseModule.
.3 bootstrap - bash script.
.3 bootstrap SQL files.
}

In order to bootstrap an ABLS instance for development or debugging purposes, one can simply run the 
following commands.

\begin{lstlisting}
$> ./DatabaseModule/bootstrap 
$> python Boostrap.py
$> python Main.py -l
\end{lstlisting}

The first bootstrap script will wipe the database files and configure them for use with an 
ABLS instance. This script should be modified if the user wants to change the physical
location of each database server. The second command will tell the Bootstrap program
to insert a set of fake data into the log, user, and audit\_user databases. This will enable
the developer to test the new ABLS instance using some predefined data. Finally, the third
command runs the {\tt Main.py} and starts the logging service (``-l'') so that new log messages
may be intercepted from a client. \newline

If the user wants to start the verification or audit services as well they can simple pass the ``-v'' or
``-a'' flags to the {\tt Main.py} program, respectively. Parameters for these services (i.e. the number of 
verification threads) can be configured by changing the source code in the respective modules
({\tt VerifyModule} and {\tt AuditModule}).


%-----------------------------------------------------------
%\addcontentsline{toc}{chapter}{\numberline{}Bibliography}
%\include{biblio}
%-----------------------------------------------------------
%\appendix
%\include{app4}
%\include{app1}
%\include{app2}
%\include{app5}
%\include{app3}
%-----------------------------------------------------------

\bibliography{ref}{}
\bibliographystyle{plain}

\end{document}
