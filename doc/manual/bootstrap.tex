\chapter{Bootstrapping}

ABLS comes packaged with a set of configuration scripts and SQL files that initialize
the database to a clean state. These files are included in the DatabaseModule directory
that comes packaged with ABLS, as shown below:

\dirtree{%
.1 ABLS.
.2 main - main executable.
.2 LoggerModule.
.2 PolicyEngineModule.
.2 AuditModule.
.2 Common.
.2 TestModule.
.2 DatabaseModule.
.3 bootstrap - bash script.
.3 bootstrap SQL files.
}

In order to configure a new ABLS instance to be run on a server in development mode, one must run
the following commands from the root ABLS directory.

\begin{lstlisting}
$> ./DatabaseModule/bootstrap 
$> python main.py -c 
\end{lstlisting}

The first bootstrap script will wipe the database files and configure them for use with an 
ABLS instance. This is the script that should be modified if the user wants to change the physical
location of each database server. The second command will tell the main ABLS executable
script to ``configure'' the database with some fake data for testing purposes. As such,
this should only be used when configuring ABLS for development tasks.

Once complete, the user should then run the following command from the root ABLS directory
to start an ABLS instance on the local host.

\begin{lstlisting}
$> python main.py -s
\end{lstlisting}

If one wants to deploy an ABLS instance in production mode, they should only run the main executable with the 
``-s'' flag, not the ``-c'' flag. Also, for convenience, these two flags can be combined during the bootstrapping
process, as shown below.

\begin{lstlisting}
$> ./DatabaseModule/bootstrap 
$> python main.py -c -s
\end{lstlisting}