\section{Bootstrapping}

ABLS comes packaged with a set of configuration scripts and SQL files that initialize
the database to a clean state. These files are included in the DatabaseModule directory
that comes packaged with ABLS, as shown below:

\dirtree{%
.1 ABLS.
.2 Main.py - main executable.
.2 Bootstrap.py - bootstrap file for the database.
.2 LoggerModule.
.2 PolicyEngineModule.
.2 AuditModule.
.2 VerifyModule.
.2 CryptoModule.
.2 Common.
.2 TestModule.
.2 DatabaseModule.
.3 bootstrap - bash script.
.3 bootstrap SQL files.
}

In order to bootstrap an ABLS instance for development or debugging purposes, one can simply run the 
following commands.

\begin{lstlisting}
$> ./DatabaseModule/bootstrap 
$> python Boostrap.py
$> python Main.py -l
\end{lstlisting}

The first bootstrap script will wipe the database files and configure them for use with an 
ABLS instance. This script should be modified if the user wants to change the physical
location of each database server. The second command will tell the Bootstrap program
to insert a set of fake data into the log, user, and audit\_user databases. This will enable
the developer to test the new ABLS instance using some predefined data. Finally, the third
command runs the {\tt Main.py} and starts the logging service (``-l'') so that new log messages
may be intercepted from a client.

If the user wants to start the verification or audit services as well they can simple pass the ``-v'' or
``-a'' flags to the {\tt Main.py} program, respectively. Parameters for these services (i.e. the number of 
verification threads) can be configured by changing the source code in the respective modules
({\tt VerificationModule} and {\tt AuditModule}).

\section{Configuration}

The network and database connectivity options for an ABLS instance are defined in the file {\tt abls.conf}, which
is located in the root directory of an ABLS system. Users can modify this file to change the network settings (i.e. 
host name, log proxy port, audit proxy port, etc) and the database connections. A snippet of a configuration file
is shown below.

\begin{lstlisting}
# Network configuration paramters
abls_host = localhost
abls_logger_port = 9998
abls_audit_port = 9999

# Database configuration string
location.db.log = ~/DatabaseModule/log.db
location.db.key = ~/DatabaseModule/key.db
location.db.users = ~/DatabaseModule/users.db
location.db.audit_users = ~/DatabaseModule/audit_users.db
location.db.policy = ~/DatabaseModule/policy.db
\end{lstlisting}

Since ABLS is in the prototype phase and does not need to be deployed to a production environment, it only 
supports local SQLite databases. Thus, the database location strings simply correspond to the names of 
local database files that are used to persist all log information used at runtime. Future versions of ABLS will 
provide the user with a more comprehensive set of database configuration options.


\section{Installation}

ABLS utilizes many third-party packages and libraries to run. For brevity, these are listed below
along with the online locations where they can be downloaded. The user is left with the task of 
installing them on their own machines in order to deploy an ABLS instance.

\begin{enumerate}
	\item Charm Crypto - \url{http://charm-crypto.com/Main.html}
	\item Pykka (Python Akka Library) - \url{http://www.pykka.org/en/latest/}
	\item SQLite - \url{http://www.sqlite.org/}
\end{enumerate}

