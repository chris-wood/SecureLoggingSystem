\section{Configuration}

The network and database connectivity options for an ABLS instance are defined in the file {\tt abls.conf}, which
is located in the root directory of an ABLS system. Users can modify this file to change the network settings (i.e. 
host name, log proxy port, audit proxy port, etc) and the database connections. An example configuration file
is shown below.

\begin{lstlisting}
# Network configuration paramters
abls_host = localhost
abls_logger_port = 9998
abls_audit_port = 9999

# Database configuration string
location.db.log = ~/DatabaseModule/log.db
location.db.key = ~/DatabaseModule/key.db
location.db.users = ~/DatabaseModule/users.db
location.db.audit_users = ~/DatabaseModule/audit_users.db
location.db.policy = ~/DatabaseModule/policy.db
\end{lstlisting}

Since ABLS is in the prototype phase and does not need to be deployed to a production environment, it only 
supports local SQLite databases. Thus, the database location strings simply correspond to the names of 
local database files that are used to persist all log information used at runtime. Future versions of ABLS will 
provide the user with a more comprehensive set of database configuration options.

